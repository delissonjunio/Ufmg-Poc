\documentclass[12pt, oneside, a4paper,english,brazil]{abntex2}

\usepackage[utf8]{inputenc}		% Codificação do documento (conversão automática dos acentos)
\usepackage{lmodern}			% Usa a fonte Latin Modern			
\usepackage[T1]{fontenc}		% Seleção de códigos de fonte.
\usepackage{lastpage}			% Usado pela Ficha catalográfica
\usepackage{indentfirst}		% Indenta o primeiro parágrafo de cada seção.
\usepackage{color}				% Controle das cores
\usepackage{graphicx,url}
\usepackage{microtype} 			% para melhorias de justificação
\usepackage{multirow}
\usepackage[super]{nth}
\usepackage[table,xcdraw]{xcolor}

\usepackage{booktabs}

\titulo{Uma plataforma para a minimização de risco em portfólios financeiros utilizando CVaR}
\autor{Délisson Junio Gonçalves Silva}
\local{Belo Horizonte, Minas Gerais}
\data{2021}
\orientador{Cristiano Arbex Valle}
\instituicao{%
 Universidade Federal de Minas Gerais 
 \par Instituto de Ciências Exatas
 \par Departamento de Ciência da Computação}
\tipotrabalho{Projeto de Pesquisa}

\preambulo{Proposta de Pesquisa Tecnológica}

\begin{document}

\imprimircapa
\imprimirfolhaderosto

\tableofcontents* %\sumario %comando que gera o sumário automaticamente

\textual

\chapter{Introdução}
De acordo com a B3\footnote{B3: \url{https://b3.com.br/}}, 2 milhões de novos investidores entraram na bolsa entre 2019 e 2020, um crescimento de mais de 200\% na base de investidores cadastrados\footnote{B3 divulga estudo sobre os 2 milhões de investidores que entraram na bolsa entre 2019 e 2020: \url{http://www.b3.com.br/pt_br/noticias/investidores.htm}. Acesso em 29/06/2021}. Desses novos investidores, 73\% obtêm informações sobre investimentos na internet e 60\% o fazem por meio de influenciadores digitais, e 30\% de todos os investidores com patrimônio abaixo de R\$ 10.000 investem em mais de cinco ativos na bolsa \footnote{A descoberta da bolsa pelo investidor brasileiro: \url{http://b3.com.br/lumis/portal/file/fileDownload.jsp?fileId=8AE490CA76400395017662491534717C}. Acesso em 29/06/2021}.

Em meio a tanta informação disponível aos investidores, se torna crucial a objetividade e o acesso a ferramentas modernas e testadas para gerência de seus portfólios. Apesar disso, esses pequenos investidores raramente têm acesso a ferramentas utilizadas por grandes empresas ou casas de investimento, o que limita muito sua capacidade de tomar decisões de forma independente.

A proposta deste trabalho é desenvolver um sistema acessível onde os usuários tenham acesso ao cálculo da perda média esperada (ou CVaR, Conditional Value At Risk) de um portfólio baseado em ativos listados na bolsa de valores brasileira.

\chapter{Referencial Teórico}
A gestão de risco é uma área de estudo e desenvolvimento que visa reduzir, controlar e conhecer os riscos associados a um certo processo ou decisão. No contexto dos mercados financeiros, isso pode ser estudado através de medidas que aproximam o downside risk (risco de perdas inesperadas), como o VaR e o CVaR. 

O VaR (Value At Risk) é uma medida da perda máxima de um investimento em um determinado período de tempo para um dado nível de confiança. Ou seja, dado um portfólio de ativos, a medida VaR a um dado nível de confiança indica qual a perda máxima que se espera, dado tal nível de confiança, em um período de tempo. Ele pode ser usado para responder, por exemplo, à pergunta "Qual a perda máxima com 95\% de certeza que meu portfólio pode acarretar nos próximos 3 meses?". Apesar de ser uma boa aproximação, essa medida tem algumas propriedades indesejadas que fazem com que seja um instrumento muito básico para tomada de decisões de otimização de portfólio, como a não-coerência e falta de subaditividade. Além disso, estudos têm demonstrado sua ineficiência na estimação de risco, particularmente em momentos de crise \cite{Mutu2011}

Já o CVaR (Conditional Value At Risk) indica qual o retorno esperado dos $\alpha\%$ piores casos de um portfólio. Proposto por [Rockafellar e Uryasev 2000]\nocite{Rockafellar2000}, o CVaR é uma medida de risco coerente e facilmente otimizado, pois é uma medida convexa. Dessa forma, é possível calcular, através de programação inteira, um portfólio ótimo para a redução de risco.

Considerando as ferramentas acima, a proposta deste trabalho se resume em desenvolver um sistema Web para a visualização das métricas VaR e CVaR acima, assim como o cálculo da otimização de alocação de ativos em um portfólio considerando a bolsa de valores brasileira. Existem vários estudos que realizam o cálculo das métricas e/ou da otimização do portfólio, como o trabalho desenvolvido por [Videira e Maciel 2019]\nocite{Maciel2019} ou [Barrozo e Lima 2019]\nocite{Gabriel2019}. Todavia, todos esses trabalhos são análises estáticas e não interativas.

\chapter{Metodologia}
O objetivo deste trabalho é desenvolver um sistema Web que permita aos usuários calcular as métricas de VaR e CVaR para um determinado portfólio e também otimizar a alocação de ativos dentro desse portfólio com respeito à CVaR, utilizando solucionadores de problemas de otimização linear. Este sistema será hospedado na Internet, disponível para utilização através de navegadores. Especificamente, serão executadas as seguintes atividades:

\begin{enumerate}
  \item Definir o dataset a ser utilizado. \\ Escolher e configurar um dataset viável que forneça cotações e/ou retornos históricos de ativos da bolsa de valores brasileira.
  \item Construir o algoritmo de cálculo de VaR e CVaR paramétrico. \\ Os algoritmos paramétricos são uma boa primeira abordagem ao tema e irão oferecer a possibilidade de validar o trabalho a ser realizado adiante.
  \item Construir o algoritmo de cálculo de VaR e CVaR histórico dado um portfólio. \\ Esses algoritmos são mais simples do que o passo de otimização e são uma forma de validar a abordagem e a escolha do dataset.
  \item Criar uma interface de usuário básica que sirva como ponto central de interação. \\ Essa interface deverá suportar as operações básicas definidas acima e estar integrada com o dataset escolhido.
  \item Implementar o cálculo de VaR e CVaR para um portfólio arbitrário na interface. \\ Cria a funcionalidade de cálculo de métricas dado um portfólio arbitrário escolhido pelo usuário e outros parâmetros relevantes ao cálculo.
  \item (POC II) Calcular a alocação ótima de investimentos nos ativos em um portfólio escolhido. \\ Implementar a otimização da métrica CVaR através da utilização de sistemas para resolução de problemas de otimização linear.
\end{enumerate}


\chapter{Resultados Esperados}

Este trabalho resulta em um sistema Web acessível publicamente pela Internet, onde investidores serão capazes de utilizar as ferramentas de análise
de risco VaR e CVaR, assim como entender o seu propósito para um portfólio balanceado. Essa ferramenta será capaz de efetuar cálculos paramétricos, baseados em distribuições definidas pelo usuário, ou históricos, onde são utilizados dados reais de mercados financeiros. Na disciplina de POC II, será adicionada a funcionalidade de otimização
de alocação de ativos dentre um portfólio, onde o usuário será capaz de executar otimizações baseadas em análise de risco interativamente.

\chapter{Etapas e Cronograma}
% Descrever o cronograma previsto para a realização dos passos definidos na Metodologia com resolução em nível de semanas.

\begin{table}[htbp]
\begin{tabular}{|l|l|l|l|l|l|l|}
\hline
\multirow{2}{*}{Atividade}                                                                                                & \multicolumn{2}{l|}{Julho} & \multicolumn{2}{l|}{Agosto} & \multicolumn{2}{l|}{Setembro} \\ \cline{2-7} 
                                                                                                                          & 1ª Q         & 2ª Q        & 1ª Q         & 2ª Q         & 1ª Q          & 2ª Q          \\ \hline
\begin{tabular}[c]{@{}l@{}}Construir o algoritmo de\\ cálculo de VaR e CVaR paramétrico\end{tabular}                      &      \cellcolor{lightgrey}        &     \cellcolor{lightgrey}        &              &              &               &               \\ \hline
\begin{tabular}[c]{@{}l@{}}Definir o dataset a ser\\ utilizado no trabalho\end{tabular}                                   &              &     \cellcolor{black}        &     \cellcolor{lightgrey}         &              &               &               \\ \hline
\begin{tabular}[c]{@{}l@{}}Construir o algoritmo de cálculo\\ de VaR e CVaR histórico dado um portfólio\end{tabular}      &              &             &     \cellcolor{lightgrey}         &      \cellcolor{lightgrey}        &               &               \\ \hline
\begin{tabular}[c]{@{}l@{}}Criar uma interface de usuário básica\\ que sirva como ponto central de interação\end{tabular} &              &             &              &       \cellcolor{lightgrey}       &               &               \\ \hline
\begin{tabular}[c]{@{}l@{}}Implementar o cálculo de VaR e CVaR\\ para um portfólio arbitrário na interface\end{tabular}   &              &             &              &     \cellcolor{lightgrey}         &      \cellcolor{lightgrey}         &               \\ \hline
\end{tabular}
\caption{Cronograma das atividades propostas neste trabalho.}

\end{table}

\postextual


\bibliographystyle{abntex2-alf} %ou abntex2-num
\bibliography{referencias}
\nocite{*}

\end{document}



